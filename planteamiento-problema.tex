%Ante los enfoques teóricos y prácticos estudiados anteriormente, nuestro desarrollo de software propio, que permite mostrar de manera gráfica las relaciones entre actores delictuales en el Sistema Penal de la Provincia del Chubut, se potencia como una herramienta vital de apoyo en la toma de decisiones de la investigación penal de bandas delictivas.
Existen antecedentes reales que justifican la importancia de esta línea de trabajo. 
En el año 2019 fue necesaria una investigación criminal sobre reiterados robos de televisores LCD en domicilios~\cite{noticiaLCDdiario}, como así también una serie de hechos consecutivos vinculados al robo de cajas fuertes en empresas del parque industrial de la ciudad de Trelew.
La UAC (Unidad de Análisis Criminal), organismo auxiliar perteneciente al MPF, sirvió como equipo de apoyo en la investigación de ambos modus operandi, haciendo uso de toda la información de los legajos fiscales, consultas generales y específicas contenidas en el Sistema Coirón. Fue de vital uso la información referida a los \textit{grupos de pertenencia} de cada persona, pero devino en un arduo trabajo entrecruzando información de personas, para dar con las supuestas bandas delictivas detrás de estos hechos.
Esta necesidad ha activado la línea de trabajo actual, utilizando la información ya contenida en el sistema de gestión penal, buscando proveer de una forma más directa y visual la base para el apoyo a la toma de decisiones en las investigaciones de bandas delictivas. 
%Dichas investigaciones sirvieron como puntapié inicial para realizar este trabajo y poder facilitar la información ya contenida en el sistema de gestión penal, de otra manera, de una forma más directa y visual a la hora de investigar, que sirva directamente como apoyo a la toma de decisiones en las investigaciones de bandas delictivas. 
Esto ayuda a los especialistas a detectar triangulaciones, transitividades y por supuesto \textit{centralidades} e importancias internas en la Red. 
%Todo ello, sumado a los indicios de investigación y la propia experticia en la temática completan una herramienta de análisis para determinar ciertas bandas o grupos altamente relacionados.
Para esto es necesaria la consideración de técnicas que permitan distinguir la \textit{importancia} de un nodo en un grafo particular, como explicamos en la siguiente sección.