Las organizaciones criminales son grupos que operan fuera de la ley, realizando actividades ilegales en beneficio propio y en detrimento de otros individuos o grupos sociales~\cite{finckenauer2005problems}. Pueden ser de diverso tamaño y cubrir áreas geográficas variadas, en muchos casos en conflicto con otras organizaciones similares. Una de las características particulares de este tipo de organizaciones es que, al estar enfocadas en actividades ilegales perseguidas por los organismos de seguridad pública, el anonimato y/o la discreción de sus miembros es de vital importancia. Esto requiere estudios de la información existente con el fin de identificar los criminales y realizar acciones apropiadas para la prevención del delito.
Los miembros de las organizaciones criminales tienen a su vez diversos grados de compromiso con cada una de ellas. En muchos casos los hechos son cometidos por individuos de baja jerarquía y responsabilidad en el grupo. Asimismo, existen otros individuos de mayor jerarquía y responsabilidad en la organización criminal, que ostentan cualidades de liderazgo. Es aquí donde nuestro trabajo puede aportar un rol significativo.