
Desde hace algunos pocos años, el Análisis de Redes Sociales (o SNA por sus siglas en inglés de Social Network Analysis) ha contribuido a las investigaciones criminales y a las actividades de inteligencia relacionadas.
Una red social modela individuos como nodos, vinculados entre sí por arcos o aristas que representan las relaciones entre esos individuos. El estudio de estas redes es importante porque se enfoca en la abstracción de las relaciones humanas sobre uno o más aspectos particulares ~\cite{pm2018practical}~\cite{burcher2020social}. De esta manera, las redes conforman estructuras de grafos en las cuales es posible identificar diversas propiedades, tales como la relevancia o la importancia relativa de los nodos individuales en función de las conexiones existentes o el flujo de información. Según Sage~\cite{scott2011sage} , existen cuatro pilares fundamentales del análisis de redes: el reconocimiento de la importancia de las relaciones sociales entre los individuos, la recolección y análisis de datos sobre estas relaciones, la importancia de la representación visual de estos datos y la necesidad de modelos matemáticos y computacionales que expliquen los patrones de conexión entre los individuos.

En particular, la vinculación entre el estudio de las redes sociales y la investigación criminal ha sido encarada por varios autores. A mediados de los 70 se utilizaban modelos básicos para establecer y cualificar las relaciones entre individuos o actores de un escenario particular, definiendo grafos de acuerdo a la información recolectada~\cite{harper1975application}, pero el procesamiento era mayoritariamente manual y con varias etapas de refinamiento y valoración de datos. Esta es la que según Klerk~\cite{Klerks1999TheNP} sería la primera generación de análisis de redes en criminalística. La segunda generación involucra el uso de herramientas computacionales que automatiza parte de la tarea de registro y estructuración de datos. Estas herramientas además aumentaron notoriamente la cantidad de datos que se pueden analizar, haciendo mucho más ágil su registro y consulta. La tercera y actual generación establece la definición de modelos y técnicas matemáticas para la generación de nuevo conocimiento, como la identificación de posiciones de poder e influencia o la calidad de potenciales testigos o informantes. Métricas como la centralidad de un nodo en un grafo son especialmente útiles en este escenario.

Uno de los trabajos más importantes al respecto es el de Krebs ~\cite{krebs2002mapping}, en donde se identifica una parte de la red de terroristas que fue responsable de los atentados del 11 de septiembre de 2001 en Nueva York. Aquí identifica agrupaciones de individuos que se conectan entre sí por los pilotos responsables del secuestro de las aeronaves. Otros estudios similares han sido efectivos en consecuencia ~\cite{medina2014social}~\cite{qin2005analyzing}~\cite{stollenwerk2016taking}.  Por otro lado, el análisis de redes sociales ha cobrado también interés en la investigación criminal tradicional como las estructuras de la mafia o el narcotráfico ~\cite{bouchard2013advances}~\cite{bright2015use}~\cite{giommoni2017illicit}~\cite{morselli2009hells}~\cite{morselli2010assessing}. Estudios como el de Malm ~\cite{malm2011networks} han permitido identificar roles en la cadena de suministros para la fabricación de drogas ilícitas, lo que acarrea diferentes riesgos penales para cada uno de los colaboradores. Otros estudios se enfocan en el uso del análisis de las redes sociales para otras actividades criminales, como el tráfico ilícito de arte ~\cite{bichler2013small}, el lavado de dinero ~\cite{colladon2017using}~\cite{soudijn2014using}, corrupción policial ~\cite{lauchs2011corrupt} y bandas juveniles ~\cite{mcgloin2005policy}~\cite{bichler2014magnetic}. Existen también líneas de investigación en la disciplina referente al cibercrimen ~\cite{decary2014information}~\cite{decary2012social}~\cite{decary2013reputation}. Es claro entonces que el análisis de redes sociales puede ser aplicado a un amplio rango de actividades criminales y ha demostrado modelar apropiadamente características propias de las organizaciones ilegales, asistiendo a la prevención del delito y al diseño de políticas adecuadas para enfrentar estas actividades.

Existen sin embargo algunas dificultades que requieren aún estudios intensivos. La cantidad de información que debe manejarse es enorme, en muchos casos con información incompleta, contradictoria y no menos frecuentemente incorrecta. Además, las relaciones humanas tradicionales se mezclan naturalmente con las interacciones ilícitas entre los individuos por lo que es necesario identificar apropiadamente su naturaleza y consecuencias y determinar los límites sensatos de la red social analizada. 

Actualmente los organismos estatales encargados de la Justicia y la prevención del delito cuentan con registros informatizados de las actividades criminales detectadas, así como de las etapas y eventos del subsecuente proceso penal. En particular, para este trabajo es de especial interés la información producida a tal efecto por las fuerzas policiales de la Provincia del Chubut y su Poder Judicial de la mano del Ministerio Público Fiscal (MPF). Existen decenas de miles de registros que son utilizados principalmente para la acción penal, pero que pueden ser empleados para modelar diferentes redes sociales sobre las cuales aplicar un análisis matemático y computacional en la búsqueda de nueva información. Esto permitirá conocer más sobre las actividades criminales y sus autores en la jurisdicción de esa provincia, con las particularidades propias de la información registrada digitalmente.

El análisis y exploración de estos grandes conjuntos de datos y sus relaciones debe ser asistido por técnicas y herramientas que faciliten este proceso y reduzcan la carga cognitiva que recae sobre los usuarios. En tal sentido, el área de Visualización de Información, en particular la Visualización de Grandes Conjuntos de Datos, busca asistir a los usuarios de tal manera. La aplicación de técnicas visuales para la representación de este tipo de información no es nueva ~\cite{xu2005criminal}~\cite{feng2019big}~\cite{mathew2021criminal}. También es importante el estudio de las tareas e interacciones que la visualización debe soportar ~\cite{chen2005visualization}, ya que son estas interacciones las que facilitan la exploración de la visualización de información.
