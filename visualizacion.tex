Para llevar a cabo la visualización del conjunto de datos obtenidos del análisis anteriormente descripto, se utilizó \texttt{Vis.js} \footnote{https://visjs.org/}, una biblioteca o librería de visualización dinámica basada en lenguaje Javascript. Esta librería está diseñada para manejar grandes cantidades de datos dinámicos y permitir la manipulación y la interacción con los datos. 
%El componente \texttt{Network} permite mostrar redes en grafos y manejar grandes cantidades de nodos. 
Aplicamos además algoritmos de diseño forzados ``\textit{force-directed graph drawing}", que intentan posicionar los nodos considerando las fuerzas entre dos nodos (atractivos si están conectados, repulsivos de lo contrario). Generalmente son iterativos y mueven los nodos uno por uno hasta que ya no es posible mejorar o se alcanza el número máximo de iteraciones. Los enlaces tienen más o menos la misma longitud y el menor número posible de enlaces cruzados. Los nodos conectados se juntan más mientras que los nodos aislados se alejan hacia los lados.
