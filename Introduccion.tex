
En la actualidad las actividades criminales habituales en una ciudad o región van desde hurtos y robos de poca importancia, hasta otros de mayor gravedad como abusos sexuales y  homicidios. Todos ellos son registrados de diferentes formas por las fuerzas de la ley, con datos de variada precisión que incluyen usualmente la tipificación del delito, los datos en tiempo y espacio, y en muchas ocasiones los autores correspondientes.
Toda esta información respalda los procesos de investigación judicial de cada caso, pero con el transcurso del tiempo constituyen una extensa base de conocimiento sobre la cual es posible extraer valiosa información para la prevención del delito y la búsqueda de la justicia. 
Por ejemplo, es posible inferir relaciones de amistad o conveniencia entre diversos autores de actividades criminales a partir de los registros delictivos y es de extrema relevancia para la prevención del delito y la resolución de casos inconclusos.
Las organizaciones criminales son grupos que operan fuera de la ley, realizando actividades ilegales en beneficio propio y en detrimento de otros individuos o grupos sociales~\cite{finckenauer2005problems}. Pueden ser de diverso tamaño y cubrir áreas geográficas variadas, en muchos casos en conflicto con otras organizaciones similares. Una de las características particulares de este tipo de organizaciones es que, al estar enfocadas en actividades ilegales perseguidas por los organismos de seguridad pública, el anonimato y/o la discreción de sus miembros es de vital importancia. Esto requiere estudios de la información existente con el fin de identificar los criminales y realizar acciones apropiadas para la prevención del delito.
Los miembros de las organizaciones criminales tienen a su vez diversos grados de compromiso con cada una de ellas. En muchos casos los hechos son cometidos por individuos de baja jerarquía y responsabilidad en el grupo. Asimismo, existen otros individuos de mayor jerarquía y responsabilidad en la organización criminal, que ostentan cualidades de liderazgo.
En tal sentido, con el objetivo de ayudar en la identificación de las bandas delictivas, sus integrantes y el grado de importancia de cada uno dentro de ellas, son de interés dos áreas de las Ciencias de la Computación: el área de Visualización de Información, en particular la Visualización de Grandes Conjuntos de Datos, que busca asistir a los usuarios en la adecuada comprensión de la información, y el Análisis de Redes Sociales, en donde se emplean técnicas y formalismos para la comprensión de las estructuras de las redes y sus nodos . 
En particular, la aplicación de técnicas visuales para la representación de este tipo de información no es nueva ~\cite{xu2005criminal}~\cite{feng2019big}~\cite{mathew2021criminal}. También es importante el estudio de las tareas e interacciones que la visualización debe soportar ~\cite{chen2005visualization}, ya que son estas interacciones las que facilitan la exploración de la visualización de información.
En este trabajo es de especial interés la aplicación de estas técnicas y tecnologías, como así también el desarrollo de un módulo de software para  la visualización de los datos, incorporando nociones de análisis de grafos, como la utilización del algoritmo de PageRank y la detección de comunidades, en pos de encontrar delincuentes de relevancia. Para esto se cuenta con los registros de actividades criminales a través de la colaboración del Ministerio Público Fiscal de la provincia del Chubut, como se detalla en la siguiente sección.
