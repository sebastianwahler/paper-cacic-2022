Como continuación de este trabajo, existen diversas líneas de investigación que quedan abiertas y en las que es posible continuar trabajando. Durante el desarrollo de este trabajo han surgido algunas líneas futuras que se han dejado abiertas y que se esperan atacar en un futuro; algunas de ellas, están más directamente relacionadas con este trabajo y son el resultado de cuestiones que han ido surgiendo durante la realización del mismo.

Actualmente se 	continúa trabajando en el desarrollo de la aplicación de visualización. Se pretende incorporar a la fórmula original de PageRank, la posibilidad de darle mayor ranking inicial a aquellos nodos que tengan un peso mayor que otros. También sería interesante hacer lo mismo con el peso que poseen los enlaces entre nodos, ya que no es lo mismo una relación de 2 casos penales en cómun entre dos personas, que una de 18 casos en común. De esta manera lograríamos darle más ranking también a aquellos nodos que estén relacionados con otros en más casos penales. También se buscará profundizar sobre diversos algoritmos de centralidad. De Intermediación (Betweenness centrality), obtiene la medida en que un nodo en particular se encuentra entre otros nodos en una red. La medida  de Cercanía (Closeness centrality), que es la inversa de la suma de los caminos más cortos (geodésicas) que conectan un nodo particular con todos los demás nodos de una red. La idea es que un delincuente es central si puede interactuar rápidamente con todos los demás, no solo con sus primeros vecinos~\cite{newman2005measure}. El algoritmo  de Vector Propio o Autovector (Eigenvector centrality), es otra forma de asignar la centralidad a un actor de la red basado en la idea de que si un nodo tiene muchos vecinos centrales, también debería ser central.
 