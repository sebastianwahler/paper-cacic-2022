Como parte simplificada de la estructura de una comunidad de nodos en una red social, cada uno representa a un individuo y la red tiene una segmentación multitudinaria ~\cite{ma2014exploring}. Algunas personas son centrales en la comunidad, algunas están al margen, establecen menos relaciones con otros y, por lo tanto, tienen una influencia menor. En esta sección, presentamos un nuevo enfoque de descubrimiento comunitario basado en el algoritmo PageRank para encontrar a estos delincuentes "importantes" ó con "mayor influencia" en nuestro grafo, a fin de encontrar supuestas bandas delictivas.

\subsubsection{Definición formal de Grafo}
Un grafo es un par $G = (N, A, g)$ donde $N$ es un conjunto finito no vacío de elementos denominados \textit{nodos} (vértices), $A$ es un conjunto de arcos y $g$ es una función que asocia a cada arco $a$ perteneciente a $A$ con un par no ordenado $(x, y)$, siendo $x$ e $y$ nodos pertenecientes a $N$. Se dice que $a$ es un arco con vértices extremos $x$ e $y$ ~\cite{dubinsky1984mathematical}.

\subsubsection{PageRank}
PageRank (PR) es un método que fue implementado a través de un algoritmo
originalmente utilizado por Google que asigna a cada página web de un conjunto dado, un puntaje que refleja su importancia dentro del conjunto. A este puntaje se lo denomina \textit{valor de PageRank}. Ante una consulta, el buscador utiliza estos puntajes para determinar el nivel de relevancia de las páginas, y retorna en primer lugar aquellas con un puntaje más alto. Para calcular los puntajes, PageRank utiliza la estructura de enlaces de la web ~\cite{brin1998anatomy}. Una página web tiene un valor de PageRank alto si es apuntada por muchas otras páginas, o bien si es apuntada por páginas con puntajes altos ~\cite{page1999pagerank}. PageRank tiene una base intuitiva en el concepto de \textit{random walks} sobre grafos ~\cite{gobel1974random}: suponga que un navegante aleatorio empieza a navegar la web desde una página cualquiera. El navegante puede hacer clic en forma aleatoria sobre alguno de los enlaces presentes en la página en la que se encuentra actualmente con una probabilidad $d$ a la que se denomina \textit{damping factor}, o bien con probabilidad $1-d$ accede aleatoriamente a cualquier otra página web. Este proceso se repite indefinidamente. Luego, el valor de PageRank de una página $P$ puede ser interpretado como la probabilidad de que el navegante aleatorio se encuentre en $P$ al finalizar el proceso. PageRank es definido formalmente de la siguiente manera ~\cite{franceschet2011pagerank}. Sean $q_i$ el número de enlaces salientes que posee la página $i$ , $n$ el número total de páginas web, $d$ el \textit{damping factor} que por lo general adquiere el valor 0.85, $\pi$ un vector columna denominado \textit{vector PageRank}, y $H = (h_{ij})$ una matriz cuadrada de tamaño $n$ tal que $h_{ij} = 1/q_i$ si existe un enlace desde la página $i$ a la página $j$ , y $h_{ij} = 0$ en caso contrario. El valor $h_{ij}$ corresponde a la probabilidad de acceder a la página $j$ desde la página $i$ en un paso, a partir de hacer clic en alguno de los enlaces que aparecen en esta última. El valor de PageRank correspondiente a la página $j$ es $\pi_j$, y se define recursivamente como se muestra en la ecuación \ref{eqn:ecuacionPageRank} ~\cite{lin2010data}.

\begin{equation} 
	\label{eqn:ecuacionPageRank} 
	\pi_j = \frac{1-d}{n} + d \sum_{i=1}^{n} \pi_i h_{ij} 
\end{equation}

\subsubsection{Aplicación de PageRank para bandas delictivas}
Nuestro dataset descrito anteriormente se obtiene a partir de consultas SQL a la Base de Datos de Coirón. Para hacer uso del algoritmo de PageRank se decidió incorporarlo dentro de esas consultas SQL de modo de obtener un resultado que pueda ser utilizado para la visualización.

