
Aquí describimos las técnicas comunes que las fuerzas de seguridad suelen utilizar para predecir posibles bandas delictivas. 

Las personas nos movemos habitualmente entre lugares conocidos o nodos (hogar, trabajo, supermercado, restaurante) y por las mismas calles o rutas. La teoría sugiere que cuando ocurre un delito es porque se cruzan delincuentes y víctimas dentro de algunas de estas zonas de actividad (nodo, ruta). A partir del análisis del lugar del delito se pueden determinar distintos tipos de víctimas y delincuentes que lo frecuentan, entender por qué concurren a ese lugar y qué hace que se encuentre la dupla delincuente-víctima. Es una manera estructurada de conocer e investigar patrones de comportamiento.

Por otro lado se puede deducir que los delincuentes se comportan igual que el resto de las personas, realizan actividades diariamente, se mueven por rutas conocidas para ir de la casa al trabajo, o a algún otro lugar que frecuenten. Es decir, mantienen una cierta rutina en sus vidas. Un delincuente tenderá a cometer un delito en algún lugar que se encuentre dentro o cerca del recorrido que realiza diariamente para trasladarse desde la casa al trabajo, del trabajo a algún lugar de recreación u otro lugar habitual.

De ambos enfoques se busca encontrar la mayor cantidad de patrones de ocurrencia entre diversos hechos de similar criminalidad y patrones horarios, como así también las zonas geográficas en donde se producen.  

La naturaleza de los vínculos de los integrantes de una banda delictiva es una variable que aporta información sobre las características y similitudes de los miembros del grupo, atendiendo a criterios concretos: vínculo familiar, cultural, de proximidad (provienen del mismo barrio), han compartido prisión, de especialización (habilidades delictivas), la experiencia u otras capacidades, y otros tipos de vínculo.
