Recordemos que en este trabajo nuestro principal interés es la identificación asistida de bandas delictivas y sus cualidades. 
Las personas nos movemos habitualmente entre lugares conocidos  (hogar, trabajo, supermercado, restaurante) y con frecuencia por las mismas calles o rutas. La teoría sugiere que muchos delitos ocurren cuando se cruzan delincuentes y víctimas dentro de algunas de estas zonas de actividad.
%Por otro lado se puede deducir que los delincuentes se comportan igual que el resto de las personas. 
Un delincuente tenderá a cometer un delito en algún lugar que se encuentre dentro o cerca del recorrido que realiza diariamente para trasladarse o su zona de movimiento habitual.
%De ambos enfoques se busca encontrar la mayor cantidad de patrones de ocurrencia entre diversos hechos de similar criminalidad y patrones horarios, como así también las zonas geográficas en donde se producen.  
La naturaleza de los vínculos de los integrantes de una banda delictiva es una variable que aporta información sobre las características y similitudes de los miembros del grupo, atendiendo a criterios concretos: vínculo familiar, cultural, de proximidad (provienen del mismo barrio), coincidencia en prisión, especialización (habilidades delictivas), la experiencia y otras capacidades y   tipos de vínculo.
